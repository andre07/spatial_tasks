\documentclass{beamer}
\usepackage[polish]{babel}
\usepackage[utf8]{inputenc}
\usepackage[T1]{fontenc}
\usetheme{Madrid}

\usepackage{lmodern}
\usepackage{textpos}

\usepackage{graphicx}
\usepackage{animate}

%\logo{\includegraphics[height=.8cm,keepaspectratio]{GfKlogo}

\title{Fay-Herriot model}   
\author{Andrzej Surma}
\date{\today} 

%\usepackage{beamerthemeshadow}
%\beamersetuncovermixins{\opaqueness<1>{25}}{\opaqueness<2->{15}}

\begin{document}

	\begin{frame}
		\titlepage
	\end{frame} 
	
	\begin{frame}{Potrzebne dane z HHP}
		\begin{itemize}
			\item{kod GUS gminy gospodarstwa}
			\item{wielkość gospodarstwa domowego}
			\item{liczba dzieci (0-14), starszych (60+) w gospodarstwie}
			\item{dochód netto gospodarstwa}
			\item{klasa wielkości miejscowości}
			\item{wiek i płeć członków rodziny}
			\item{typ budynku}
			\item{modelowana wartość ważona kryteriami panelowymi}
		\end{itemize}
	\end{frame}
	
	\begin{frame}{Potrzebne dane z HHP}
		\begin{block}{}
			Do kalibracji wyników, niezależnie potrzebujemy informacji o raportowanych oficjalnie wartościach na poziomie województwa
		\end{block}
	\end{frame}
	
	\begin{frame}{Krok I}
		\begin{block}{Selekcja zmiennych}
			\begin{itemize}
				\item{analiza korelacji}
				\item{analiza graficzna np. boxplot}
				\item{analiza regresji}
			\end{itemize}
		\end{block}
		\begin{block}{OLS}
			Zestaw zmiennych różnicujących średnie wydatki gospodarstwa, penetrację
		\end{block}
	\end{frame}
	
	\begin{frame}{Krok II}
		\begin{block}{Dane geo o gminach}
		\end{block}
		\begin{block}{Uwagi}
			\begin{itemize}
				\item{w geo: kody GUS ulegają aktualizacji każdego roku}
				\item{w panelu: brak aktualizacji, kody niefunkcjonujące}
			\end{itemize}
		\end{block}
		\begin{block}{Macierz dystansów}
			poszukujemy podobnych gmin pod względem cech, które różnicują średnie wydatki gospodarstwa, penetrację
		\end{block}
		\begin{block}{Sąsiedztwo gmin i powiatów}
			geograficzne ujęcie bliskości
		\end{block}
	\end{frame}
	
	\begin{frame}{Wiązkowanie gmin}
			\animategraphics[controls,width=\linewidth]{1}{mapki/graph_}{1}{8}
	\end{frame}
	
	\begin{frame}{Wiązkowanie gmin}
		\begin{block}{}
			\begin{itemize}
				\item{dla każdej gminy definiujemy dla niej wiązkę gmin podobnych, która może zrzeszać podobne pod względem zdefiniowanych cech gminy z sąsiednich powiatów. Nie jest wymagane, aby wiązka była 'łukowo spójna', tzn. aby można było połączyć gminy z wiązki linią}
				\item{grupa gmin podobnych - ważona metryka Gowera; wagi proporcjonalne do wyników metody selekcji zmiennych; Q3 dystansów jako cutpoint}
				\item{podobieństwo - macierz odległości oraz sąsiedztwo geograficzne}
				\item{uwierzytelnienie estymatora średniej wartości wydatków}
				\item{brak możliwości wnioskowania na poziomie gmin na podstawie HHP}
				\item{średnia oraz wariancja wartości wewnątrz wiązki}
			\end{itemize}
		\end{block}
	\end{frame}

	\begin{frame}{Model Faya-Herriota}
		\begin{block}{}
			\begin{displaymath}
				y_{i} = x_{i}^T\beta + v_{i} + e_{i}
			\end{displaymath}

			\begin{displaymath}
				\hat{\mu}_{i} = \hat{\gamma}_{i}\hat{y}_{i} + (1-\hat{\gamma}_{i})\bar{X}_{i}^T\hat{\beta}
			\end{displaymath}
			
			\begin{itemize}
				\item{i = 1,...,n - analizowane obszary - gminy}
				\item{x - cechy wyjaśniające; y - zmienna wyjaśniana}
				\item{v - element losowy na poziomie obszaru}
				\item{$\beta$ - estymowane współczynniki}
				\item{$\gamma$ - komponent wariancyjny z modelu FH}
			\end{itemize}
			
		\end{block}		

		\begin{block}{sae::eblupFH}
			\small{Modelowanie średniej wartości zakupów per gospodarstwo w gminie.}
			\footnotesize{Estymacja współczynników modelu odbywa się tylko na części gmin, dla których liczba raportujących	gospodarstw w odpowiednich wiązkach jest >= 30.}
		\end{block}
	\end{frame}

	\begin{frame}{Kalibracja wyników}
		\begin{block}{Ostatnia faza}
			Za pomocą ważenia kwadratowego kalibracja wyników do wartości oficjalnie raportowanych przez HHP
		\end{block}
	\end{frame}
	
\end{document}
